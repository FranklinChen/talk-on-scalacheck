\documentclass{beamer}

\mode<presentation>

\usepackage{minted}

% End up with magenta links; don't know how to change color; linkcolor fails
\hypersetup{colorlinks=true}

\title{Beyond \texttt{xUnit} example-based testing: property-based testing with \href{https://github.com/rickynils/scalacheck}{\texttt{ScalaCheck}}}
\author{\href{http://franklinchen.com/}{Franklin Chen}}
\date{\href{http://www.meetup.com/Pittsburgh-Scala-Meetup/events/108080782}{Pittsburgh Scala Meetup} \\
April 11, 2013
}

\begin{document}

\begin{frame}
  \titlepage
\end{frame}

% TODO
%\section*{Outline}
%\begin{frame}
%  \frametitle{Outline}
%  \tableofcontents[pausesections]
%\end{frame}

\begin{frame}
  \frametitle{Our goal: software correctness}

  \uncover<1-> Use all available tools that are \emph{practical} in \emph{context}.

  \begin{itemize}
    \item<2-> Static \href{http://en.wikipedia.org/wiki/Type\_system}{type system}
      \begin{itemize}
        \item<3-> We are \href{http://www.scala-lang.org/}{Scala} users!
      \end{itemize}
    \item<4-> Two extremes
      \begin{itemize}
        \item<5-> \href{http://en.wikipedia.org/wiki/Automated\_theorem\_proving}{Automated theorem proving}
        \item<6-> Testing
      \end{itemize}
  \end{itemize}

  \uncover<7->{What is \emph{practical}?}
\end{frame}

% Uncover is problematic
\begin{frame}[fragile]
  \frametitle{Theorem proving}

  \begin{block}{Example: list append}
    \begin{minted}{scala}
// Given an implementation in ListOps of:
def append[A](xs: List[A], ys: List[A]): List[A]
    \end{minted}
  \end{block}

  \begin{block}{How prove this property \emph{without a shadow of a doubt}?}
    \begin{minted}{scala}
// for all lists xs and ys
ListOps.append(xs, ys).length == xs.length + y.length
    \end{minted}
  \end{block}

  \begin{block}{Not (yet) \emph{practical} in most contexts.}
    (Scala compiler cannot prove this property.)
  \end{block}
\end{frame}

\begin{frame}[fragile]
  \frametitle{More powerful static type system: \href{http://en.wikipedia.org/wiki/Dependent\_type}{dependent types}}

  \begin{block}{Try using a more powerful language?}
    \begin{itemize}
      \item Example: \href{http://idris-lang.org/}{Idris} language, based on \href{http://haskell.org/}{Haskell}
    \end{itemize}
  \end{block}

  \begin{block}{Type checks!}
    % Pretend to Pygments that Idris is Haskell
    \begin{minted}{haskell}
(++) : Vect A n -> Vect A m -> Vect A (n + m)
(++) Nil       ys = ys
(++) (x :: xs) ys = x :: xs ++ ys
    \end{minted}
  \end{block}

  \begin{block}{Still research, however.}
    \begin{itemize}
      \item Also: Scala libraries such as \href{https://github.com/milessabin/shapeless}{\texttt{shapeless}} pushing the edge of Scala's type system.
      \item Not (yet) \emph{practical} for most of us.
    \end{itemize}
  \end{block}
\end{frame}

\begin{frame}
  \frametitle{Testing}

  \uncover<1->{Goals?}

  \begin{itemize}
    \item<2-> Give up goal of mathematical guarantee of correctness.
    \item<3-> Try to gain \emph{confidence} that our code might be ``probably'' or ``mostly'' correct.
    \item<4-> Still a young field. Recent report on the state of testing: \href{http://blog.sei.cmu.edu/post.cfm/common-testing-problems-pitfalls-to-prevent-and-mitigate}{SEI blog}
  \end{itemize}

  \uncover<5->{Remember to be humble when testing; avoid overconfidence!}
\end{frame}

\begin{frame}
  \frametitle{Example-based testing}

  \begin{itemize}
    \item \texttt{xUnit} frameworks such as \href{http://junit.org/}{\texttt{JUnit}}
    \item Hand-craft specific example scenarios, make assertions
  \end{itemize}

  \uncover<1->{Popular Scala test frameworks that support \texttt{xUnit} style testing:}

  \begin{itemize}
    \item<2-> \href{http://scalatest.org/}{\texttt{ScalaTest}}
    \item<3-> \href{http://etorreborre.github.io/specs2/}{\texttt{specs2}}
  \end{itemize}

  \uncover<4->{For concreteness: I test with \texttt{specs2}, with \href{http://www.scala-sbt.org/}{\texttt{SBT}}.}
\end{frame}

\begin{frame}[fragile]
  \frametitle{Individual examples of list append}

  \begin{minted}{scala}
// Individually hand-crafted examples
ListOps.append(List(3), List(7, 2)).length must_==
  List(3).length + List(7, 2).length
ListOps.append(List(3, 4), List(7, 2)).length must_==
  List(3, 4).length + List(7, 2).length
// ...
  \end{minted}

  \begin{itemize}
    \item Tedious to set up each example, one by one.
    \item Are we confident that we listed all the relevant cases, including corner cases?
  \end{itemize}
\end{frame}

\begin{frame}[fragile]
  \frametitle{Refactoring to a data table}

  \begin{minted}{scala}
// Hand-crafted table of data, but logic is factored out
"xs"       | "ys"       |
List(3)    ! List(7, 2) |
List(3, 4) ! List(7, 2) |> {
  (xs, ys) =>
  ListOps.append(xs, ys).length must_==
    xs.length + ys.length
}
  \end{minted}

  Note: latest \texttt{JUnit} supports \href{https://github.com/junit-team/junit/wiki/Parameterized-tests}{parameterized tests} also.
\end{frame}

\begin{frame}[fragile]
  \frametitle{Property-based testing: introducing \texttt{ScalaCheck}}

  \begin{itemize}
    \item \texttt{ScalaCheck} library
      \begin{itemize}
        \item Port of Haskell \href{http://en.wikipedia.org/wiki/QuickCheck}{\texttt{QuickCheck}}
        \item Use standalone, or within \texttt{ScalaTest} or \texttt{specs2}
      \end{itemize}
    \item Write down what we really intend, a universal property!
  \end{itemize}

  \begin{minted}{scala}
prop {
  (xs: List[Int], ys: List[Int]) =>
  ListOps.append(xs, ys).length must_==
    xs.length + ys.length
}
  \end{minted}

  \texttt{ScalaCheck} automatically generates 100 random test cases (the default number) and runs them successfully!

\end{frame}

\begin{frame}
  \frametitle{Property-based testing: fake theorem proving?}

  \begin{itemize}
    \item<1-> Theorem proving is hard.
    \item<2-> \texttt{ScalaCheck} allows us to pretend we are theorem proving, with caveats:
      \begin{itemize}
        \item<3-> How random are the generated data?
        \item<4-> How useful for our desired corner cases?
        \item<5-> How can we customize the generation?
        \item<6-> How reproducible is the test run?
        \item<7-> How confident can we be in the coverage?
      \end{itemize}
  \end{itemize}

  \uncover<8->{Property-based testing is still an area of research.}

  \uncover<9->{But it is already useful as it is.}
\end{frame}

\begin{frame}[fragile]
  \frametitle{A failing test reports a minimal counter-example}

  What is wrong with this property about multiplication and division?

  \begin{minted}{scala}
prop {
  (x: Int, y: Int) => (x * y) / y must_== x
}
  \end{minted}

  Output:

  \begin{minted}{console}
> test
ArithmeticException: A counter-example is [0, 0]:
  java.lang.ArithmeticException: / by zero
  \end{minted}

  Oops, our test was sloppy!
\end{frame}

\begin{frame}[fragile]
  \frametitle{Introducing conditional properties}

  Exclude dividing by 0: use \emph{conditional} operator \texttt{==>} with \texttt{forAll}:

  \begin{minted}{scala}
prop {
  x: Int =>
  forAll {
    y: Int =>
    (y != 0) ==> { (x * y) / y must_== x  }
  }
}
  \end{minted}

  \begin{itemize}
    \item Generate random \texttt{x}.
    \item Generate random \texttt{y}, but discard test case if conditions fails.
  \end{itemize}

  All good now?
\end{frame}

\begin{frame}[fragile]
  \frametitle{Testing as an iterative process}

  \begin{itemize}
    \item Still get a counter-example.
    \item \texttt{Int} overflow.
  \end{itemize}

  \begin{minted}{console}
> test
A counter-example is [2, 1073741824]
  (after 2 tries - shrinked
   ('452994647' -> '2','-2147483648' -> '1073741824'))
'-2' is not equal to '2'
  \end{minted}

  Writing and refining a property guides our understanding of the problem.
\end{frame}

\begin{frame}
  \frametitle{The joy of experimental testing}

  \texttt{ScalaCheck} discovered an unexpected edge case for us!

  \begin{itemize}
    \item<1-> Write a general property before coding.
    \item<2-> Write the code.
    \item<3-> \texttt{ScalaCheck} gives counter-example.
    \item<4-> Choice:
      \begin{itemize}
        \item<5-> Fix code.
        \item<5-> Fix property.
      \end{itemize}
    \item<6-> Run \texttt{ScalaCheck} again until test passes.
  \end{itemize}
\end{frame}

\begin{frame}[fragile]
  \frametitle{Choice 1: fix code}

  Assume we want our high-level property, of multiplying and dividing of integers, to hold.

  Choose a different representation of integer in our application: replace \texttt{Int} with \texttt{BigInt}:

  \begin{minted}{scala}
prop {
  x: BigInt =>
  forAll {
    y: BigInt =>
    (y != 0) ==> { (x * y) / y must_== x  }
  }
}
  \end{minted}

  Success!
\end{frame}

\begin{frame}[fragile]
  \frametitle{Choice 2: fix property}

  Assume we only care about a limited range of \texttt{Int}.

  Use a \emph{generator} \texttt{Gen.choose}, for example:

  \begin{minted}{scala}
forAll(Gen.choose(0, 10000), Gen.choose(1, 10000)) {
  (x: Int, y: Int) => { (x * y) / y must_== x }
}
  \end{minted}

  Success!
\end{frame}

\begin{frame}[fragile]
  \frametitle{Introduction to custom generators: why?}

  Attempted property:

  \begin{minted}{scala}
prop {
  (x: Int, y: Int, z: Int) =>
  (x < y && y < z) ==> x < z
}
  \end{minted}

  Output:

  \begin{minted}{console}
> test
Gave up after only 28 passed tests.
  142 tests were discarded.
  \end{minted}

  Reason: too many \texttt{x}, \texttt{y}, \texttt{z} test cases that did not satisfy the condition.
\end{frame}

\begin{frame}[fragile]
  \frametitle{Introduction to custom generators: how?}

  Try to generate data likely to be useful for the property.

  Use \texttt{arbitrary} generator, \texttt{Gen.choose} generator, and \texttt{for}-comprehension:

  \begin{minted}{scala}
val orderedTriples = for {
  x <- arbitrary[Int]
  y <- Gen.choose(x + 1, Int.MaxValue)
  z <- Gen.choose(y + 1, Int.MaxValue)
} yield (x, y, z)

forAll(orderedTriples) {
  case (x, y, z) =>
  (x < y && y < z) ==> x < z
}
  \end{minted}

  Success!
\end{frame}

\begin{frame}[fragile]
  \frametitle{More complex custom generator: sorted lists}

  Assume we want to specify the behavior of a function that inserts an integer into a sorted list to return a new sorted list:
  \begin{minted}{scala}
def insert(x: Int, xs: List[Int]): List[Int]
  \end{minted}
  
  We first try
  \begin{minted}{scala}
prop {
  (x: Int, xs: List[Int]) =>
  isSorted(xs) ==> isSorted(insert(x, xs))
}
  \end{minted}

  where we have already defined
  \begin{minted}{scala}
def isSorted(xs: List[Int]): Boolean
  \end{minted}

\end{frame}

\begin{frame}[fragile]
  \frametitle{Not constrained enough}

  Too many generated lists are not sorted.

  \begin{minted}{console}
> test-only *ListCheckSpec
Gave up after only 10 passed tests.
  91 tests were discarded.
  \end{minted}

  So let's write a custom generator \texttt{someSortedLists}, to use with the property
  \begin{minted}{scala}
forAll(someSortedLists) {
  xs: List[Int] => prop {
    x: Int =>
    isSorted(xs) ==> isSorted(insert(x, xs))
  }
}
  \end{minted}
\end{frame}

\begin{frame}[fragile]
  \frametitle{Custom generator for sorted lists}

  \begin{itemize}
    \item Choose a list size.
    \item Choose a starting integer \texttt{x}.
    \item Choose a list with first element at least \texttt{x}.
  \end{itemize}

  \begin{minted}{scala}
val someSortedLists = for {
  size <- Gen.choose(0, 1000)
  x <- arbitrary[Int]
  xs <- sortedListsFromAtLeast(size, x)
} yield xs
  \end{minted}

  Now implement our \texttt{sortedListsFromAtLeast}.
\end{frame}

\begin{frame}[fragile]
  \frametitle{Generating the sorted list}

  \begin{minted}{scala}
/**
  @return generator of a sorted list of length size
          with first element >= x
 */
def sortedListsFromAtLeast(size: Int, x: Int):
    Gen[List[Int]] = {
  if (size == 0) {
    Nil
  }
  else {
    for {
      y <- Gen.choose(x, x+100)
      ys <- sortedListsFromAtLeast(size-1, y)
    } yield y::ys
  }
}
  \end{minted}
\end{frame}

\begin{frame}[fragile]
  \frametitle{What is going on? Classifiers}

  Gather statistics about the nature of the generated data. One way: classifiers.

  \begin{minted}{scala}
forAll(someSortedLists) {
  xs: List[Int] => classify(xs.length < 300,
                            "length 0-299") {
    classify(xs.length >= 300 && xs.length < 800,
             "length 300-799") {
      classify(xs.length >= 800,
               "length 800+") {
        prop {
          x: Int =>
          isSorted(xs) ==> isSorted(insert(x, xs))
        }
      }
    }
  }
}
  \end{minted}
\end{frame}

\begin{frame}[fragile]
  \frametitle{Classifier output}

  Output:
  \begin{minted}{console}
> test-only *ListCheckSpec
[info] > Collected test data: 
[info] 42% length 300-799
[info] 31% length 0-299
[info] 27% length 800+
  \end{minted}
\end{frame}

\begin{frame}[fragile]
  \frametitle{Much more!}

  Examples not covered today.

  \begin{itemize}
    \item<1-> Built-in support for generating sized containers.
    \item<2-> Specify custom frequencies when choosing alternatives.
    \item<3-> Generate objects such as trees.
    \item<4-> Generate functions.
    \item<5-> Supply own random number generator.
    \item<6-> Stateful testing.
  \end{itemize}
\end{frame}

\begin{frame}
  \frametitle{Warning: false sense of security}

  \begin{itemize}
    \item Automated testing: good
    \item Lots and lots of tests: good?
    \item Fallacy of \href{http://radar.oreilly.com/2013/04/data-skepticism.html}{big data overoptimism}
  \end{itemize}

  Testing is not proof: \emph{absence of evidence} of bugs does not equal \emph{evidence of absence} of bugs.
\end{frame}

\begin{frame}
  \frametitle{Other property-based testing tools}

  \texttt{ScalaCheck} has known limitations:
  \begin{itemize}
    \item Not always easy to write good generator.
    \item Random generation does not provide complete coverage.
    \item Tests do not give the same results when run again.
  \end{itemize}

  Alternatives:
  \begin{itemize}
    \item \href{http://www.cs.york.ac.uk/fp/smallcheck/}{SmallCheck} for Haskell
      \begin{itemize}
        \item Apparently \href{https://github.com/dwhjames/smallcheck4scala}{ported to Scala}
      \end{itemize}
    \item Active area of research
  \end{itemize}
\end{frame}

\begin{frame}
  \frametitle{The TDD community}
  
  \begin{itemize}
    \item \texttt{JUnit} incorporating some property-based ideas in the new feature called \href{https://github.com/junit-team/junit/wiki/Theories}{Theories}
    \item Many other programming languages gaining \href{http://en.wikipedia.org/wiki/QuickCheck}{property-based testing}!
    \item \href{http://www.natpryce.com/}{Nat Pryce} of \href{http://www.growing-object-oriented-software.com/}{Growing Obect-Oriented Software Guided by Tests} giving workshops on property-based TDD
  \end{itemize}
\end{frame}

\begin{frame}
  \frametitle{Conclusion}

  \begin{itemize}
    \item Automated testing is good.
    \item Property-based testing: put into your toolbox.
      \begin{itemize}
        \item If you use Scala: use \texttt{ScalaCheck}.
        \item If you use Java: use \texttt{ScalaCheck}!
      \end{itemize}
    \item Be aware of limitations of testing.
    \item Have fun!
  \end{itemize}
\end{frame}

\end{document}
